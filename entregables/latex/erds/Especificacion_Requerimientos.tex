\documentclass[12pt,letterpaper]{article}

\usepackage[utf8]{inputenc}
\usepackage[spanish]{} %Paquete para el idioma del archivo.
\usepackage[left=2.54cm,right=2.54cm,top=2.54cm,bottom=2.54cm]{geometry} %Se supone que es el margen del archivo.
\usepackage{amsmath}
\usepackage{amssymb} %Paquete paa símbolos, entre ellos las negaciones.
\usepackage{amsfonts} %Paquete para denotar letras, como conjuntos de números. 
\usepackage{graphicx}
\usepackage[all]{xy}
\usepackage{tikz}
\usepackage[normalem]{ulem}
\usepackage{soul}
\pagestyle{empty}
\usepackage{mathrsfs}
\usepackage{ upgreek }
\usepackage[pdftex]{hyperref}
\usepackage{polynom} %Paquete para escribir divisiones de polinomios.


\usepackage{fancyhdr} %Paquete para modificar pie de página.
\pagestyle{fancy} %El pie de página comienza desde ésta página.

\begin{document} 

\begin{center}
  {\LARGE \scshape Universidad Nacional Aut\'onoma de M\'exico \vspace{10mm} \\ Facultad de Ciencias }
  \rule{0.8\textwidth}{.8pt}\\
\end{center}

\subsection*{Nombre del equipo: Ilakech Alakech}
\subsection*{Sistema: PUMA eventos} 
\subsection*{Iteraci\'on: Segunda} 

\begin{center}
  {\LARGE \textbf {Especificaci\'on de Requerimientos de Software}}
\end{center} 

\subsection*{Contenido}
\smallskip \underline{Enunciado del problema.}
\\ 
\smallskip \underline{Diagramas de caso de uso} de la iteraci\'on 
\\
\smallskip \underline{Golsario de t\'erminos.}
\\
\smallskip Detalle de casos de uso de la primera iteraci\'on.

\begin{enumerate}
\item Caso de uso: Visitante \\
\end{enumerate}

\begin{center}
  \includegraphics[scale=.2]{Visitante.png}
\end{center}

Descripcion: El visitante puede ver todos los eventos que la UNAM tenga disponibles, puede ver detalles como fechas, etc., pero no puede buscar eventos ni solicitar invitaci\'on a alguno de ellos, para eso debe de estar registrado.

\begin{table}[htbp]
  \centering
  \begin{tabular}{| c | c | }
    \hline
    Paso & Acci\'on  \\ \hline
    1 & El visitante entra al sistema &  \hline
    
    2 &  El visitante da click sobre el bot\'on de eventos.\\ \hline

    3 & El sistema muestra la lista de los eventos disponibles & \hline

    4 & El visitante puede ver la lista de eventos.\\ \hline    
  \end{tabular}
  \caption{Flujo normal de eventos.}
  \label{tabla:eventos}
\end{table}

\cfoot{} % quitar número de página del centro.  

\rfoot{

  Elaboró: Miriam Torres Bucio.

  Versión: 2.0


  Fecha de elaboraci\'on:} %Pie de página izquierdo.


\end{document}
