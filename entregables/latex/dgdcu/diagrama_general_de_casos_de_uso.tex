\documentclass{article}
\usepackage[utf8]{inputenc}
\usepackage[spanish]{babel}
\usepackage{fancyhdr}
\usepackage{graphicx}
\usepackage{anysize} %Margen
\usepackage{ragged2e}

\renewcommand{\headrulewidth}{0pt}
\newcommand\tab[1][1cm]{\hspace*{#1}}
\newcommand\fecha[3]{\textit{#1/#2/#3}}
\setcounter{secnumdepth}{0}

\pagestyle{fancy}
\fancyhf{}
\fancyhead[L]{Equipo:\tab Ilakech Alakech\\Sistema:\tab[.9cm]PUMA Eventos\\Iteración:\tab[.75cm]Primera\\}
\fancyhead[R]{\includegraphics[scale=.1]{../imagenes/logo.jpg}}
\fancyfoot[L]{Elaboró:\tab[3cm]Emiliano Galeana Araujo\\Fecha de elaboración:\tab \today\\Versión: \tab[2.8cm] 1.0}

\begin{document}

%% \tableofcontents

\marginsize{2cm}{2cm}{1cm}{5cm} 

\begin{center}
  {\LARGE \scshape Plan de Proyecto\vspace{10mm} }
\end{center}

\section{Contenido}
\\
Nombre del proyecto\\
\indent Objetivo del proyecto\\
\indent Establecer fechas de inicio y fin del proyecto\\
\indent Entregables\\
\indent Fecha y forma de entrega del producto\\
\indent Identificación de las funcionalidades del producto de software
\newpage


\section{Nombre del proyecto: Primer Proyecto de Software}
\section{Objetivos del proyecto:}
Crear un producto de software cuyos objetivos son los siguientes:\\
Un sistema que integre todos los eventos organizador por la Universidad para
facilitar su búsqueda de acuerdo al interés de cada estudiante, y que permita
facilitar el manejo de la asistencia en cada uno de estos.\\
\rule{1\textwidth}{.8pt}\\

\section{Fechas}
Fecha de inicio del proyecto:\hspace{.64cm} \fecha{05}{08}{2019}\\
\indent Fecha de fin del proyecto: \hspace{1cm}\fecha{02}{12}{2019}\\
\rule{1\textwidth}{.8pt}\\

\section{Entregables}
\begin{itemize}
\item Plantilla del Primer Proyecto de Software llenada con datos del equipo.
\item Carátula de platillas personalizadas.
\item Plan del Primer Proyecto de Software siguiendo la plantilla
  \begin{itemize}
  \item Diagrama general de casos de uso.
  \item Creación del Tablero de la primera iteración.
  \end{itemize}
\item Espeficicación de Requerimientos de Software de la iteración siguiendo la
  plantilla:
  \begin{itemize}
  \item Diagrama general de casos de uso para la iteración.
  \item Detalle de casos de uso.
  \item Casos de prueba para los casos de uso.
  \item Prototipo de interfaz.
  \item Requerimientos no funcionales.
  \end{itemize}
\item Tablero de la primera iteración.
\item Especificación de Requerimientos de Software de la iteración revisado.
\item Especificación de Diseño de Software siguiendo la plantilla:
  \begin{itemize}
  \item Arquitectura del software.
  \item Diagrama de paquetes.
  \item Ambiente de implementación.
  \item Diagrama de distribución.
  \item Diagrama de clases.
  \item Diagrama de navegación.
  \item Diagrama de secuencia.
  \item Diseño de base de datos.
  \end{itemize}
\item Código de clases.
\item Documento de Construcción siguiendo la plantilla.
\item Código de clases.
\item Código de clases probado.
\item Casos de uso integrados.
\item Documento de Integración y Prueba del Producto de Software siguiendo la
  plantilla.
\item Entrega del Producto de Software.
\item Retrospectiva de la iteración siguiendo la plantilla.
\item Especificación de Requerimientos y Diseño de la 2$^{a}$ iteración.
\item Código de la 2$^{a}$ iteración.
\item Entrega de la 2$^{a}$ iteración.
\item Retrospectiva de la 2$^{a}$ iteración.
\end{itemize}
\rule{1\textwidth}{.8pt}\\

\section{Fecha y forma de entrega final del producto de software}
La entrega final se realizará el lunes 02 (dos) de Diciembre de 2019 a la hora de
clase.
\section{Identificación de las funcionalidades del producto de software}
Diagrama general de casos de uso\\
Meter el diagrama
\end{document}

